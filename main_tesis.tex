%%%%%%%%%%%%%%%%%%%%%%%%%%%%%%%%%%%%%%%%%%%%%%%%%%%%%%%%%%%%%%%%%%%%%%%%%%%%%%%%%%
% \documentclass[12pt,papel,twoside]{ibtesis}
\documentclass[12pt,screen,twoside,pagebackref]{ibtesis}
% \documentclass[12pt,papel,singlespace,oneside]{ibtesis}
% \documentclass[12pt,papel,preprint,singlespace,oneside]{ibtesis}


%%%%%%%%%%%%%%%%%%%%% Paquetes extra %%%%%%%%%%%%%%%%%%%%%%%%%%%%%%%%%%%%%%%%%%%
% Por conveniencia: aqu\'{\i} puede cargar todos los paquetes y definir los comandos 
% que necesite
\usepackage{ibextra}
%\usepackage[spanish,es-nodecimaldot]{babel}
\usepackage{amsfonts}
\usepackage{algpseudocode}
\usepackage{algorithm}
\usepackage{graphicx}
\usepackage{bm}
\usepackage{booktabs}
\usepackage{amssymb}

%\includeonly{marco teorico}
%%%%%%%%%%%%%%%%%%%%%%%%%%%%%%%%%%%%%%%%%%%%%%%%%%%%%%%%%%%%%%%%%%%%%%%%%%%%%%%%

\renewcommand{\algorithmicrequire}{\textbf{Input:}}
\renewcommand{\algorithmicensure}{\textbf{Output:}}
\floatname{algorithm}{Algoritmo}

%%%%%%%%%%%%%%%%%%%%% Informacion sobre la tesis %%%%%%%%%%%%%%%%%%%%%%%%%%%%%%%
\title{Optimización de hiperparámetros en modelos reducidos adaptativos, con aplicaciones a ondas gravitacionales y el proyecto LIGO}
\author{Atuel E. Villegas A.}
\director{Dr. Manuel Tiglio}
\codirector{Dr. Carlos Figueroa}
\carrera{Tesis de Licenciatura en F\'{\i}sica}
\grado{ }
\laboratorio{ }
\jurado{Dr.~J.~J.~Jurado (Instituto Balseiro) \\ 
Dr.~Segundo Jurado (Universidad Nacional de Cuyo)\\ 
Dr.~J.~Otro Jurado (Univ. Nac. de LaCalle)\\
Dr.~J.~L\'{o}pez Jurado (Univ. Nac. de Mar del Plata)\\
Dr.~U.~Amigo (Instituto Balseiro, Centro At\'{o}mico Bariloche)}
\palabrasclave{Ondas Gravitacionales, Bases Reducidas, Optimización de Hiperparámetros}
\keywords{Gravitational Waves, Reduced Basis, Hyperparameter Optimization}
% Si queremos poner la fecha manualmente:
% \date{Diciembre de 2099}

%%%%%%%%%%%%%%%%%%%%%%%%%%%%%%%%%%%%%%%%%%%%%%%%%%%%%%%%%%%%%%%%%%%%%%%%%%%%%%%%
%\titlepagefalse % Si no quiere compilar la portada descomente esta linea
%\includeonly{apendices} % Compilar s\'{o}lo estos archivos 
\graphicspath{{figs/}} % Lugar donde encontrar las figuras generales (se puede poner uno en cada cap{\'{\i}}tulo)
%%%%%%%%%%%%%%%%%%%%%%%%%%%%%%%%%%%%%%%%%%%%%%%%%%%%%%%%%%%%%%%%%%%%%%%%%%%%%%%%


\begin{document}

% Dentro del environment 'preliminary' va:
% la dedicatoria, resumen, abstract, indices

\begin{preliminary}

% Escriba su dedicatoria
\dedicatoria{
A mi familia\\
A mis amigos\\
A mis compañeros´
}

%%% \'{I}ndices %%%%

%\begin{abreviaturas}
                                %Abreviaturas
%\end{abreviaturas}

\tableofcontents                %\'{I}ndice

%\listoffigures                  %Figuras

%\listoftables                   %Tablas

\begin{resumen}%


La inferencia de parámetros para una colisión binaria de agujeros negros es un área de gran importancia dentro de la ciencia de ondas gravitacionales, sobre todo con el trabajo conjunto de los interferómetros LIGO, VIRGO y KAGRA generando una gran cantidad de datos para ser analizados. El estándar actual para realizar esta inferencia requiere la producción de funciones de onda en tiempo real, lo que no es posible utilizando relatividad numérica. Los modelos sustitutos de orden reducido son una gran alternativa que permite generar resultados precisos en el orden de los milisegundos.

En esta tesis se trató el problema de la optimización de hiperparámetros para un sistema de aprendizaje supervisado, el cual consiste en la primera etapa de la construcción de un modelo sustituto. Este sistema de aprendizaje es un refinamiento del método de las bases reducidas, ya utilizado en la construcción de modelos sustitutos. 

Para la optimización se utilizaron métodos bayesianos, los cuales se vienen utilizando bastante dentro de la ciencia de datos en los últimos años debido a la necesidad de crear modelos cada vez más complejos y precisos. Esta optimización se realizó utilizando la librería llamada Optuna escrita en Python. Se comparó el método con la búsqueda aleatoria y se obtuvieron buenos resultados, mostrando una clara superioridad de la optimización bayesiana, por lo menos en este contexto.


\end{resumen}

\begin{abstract}%
Parameter inference for binary black hole collisions is an area of great importance within gravitational wave science, especially with the joint work of the LIGO, VIRGO and KAGRA interferometers generating a large amount of data to be analyzed. The current standard for performing this inference requires the production of wave functions in real time, which is not possible using numerical relativity methods. Reduced order surrogate models are a great alternative that allows the generation of accurate results in the order of milliseconds.

In this thesis the problem of hyperparameter optimization for a supervised learning system is treated, which consists of the first stage of the construction of a surrogate model. This learning system is a refinement of the reduced basis method, already used in the construction of surrogate models. 

Bayesian methods were used for the optimization, which have been widely used in data science in recent years due to the need to create increasingly complex and accurate models. This optimization was performed using the Optuna library written in Python. The method was compared with random search and good results were obtained, showing a clear superiority of Bayesian optimization, at least in this context.

\end{abstract}


%%% Local Variables: 
%%% mode: latex
%%% TeX-master: "template"
%%% End: 


\end{preliminary}


% Podemos usar cualquiera de los dos comandos: \input o \include para incluir el texto
% capitulo 1
\include{introduccion}
% Capitulo 2
%\chapter{Marco Teórico}

\section{Bases Reducidas}

\section{Optimización Bayesiana} % pasé esto a hp greedy por un fallo al cargar las citas. Muy raro todo pero funcionó
\section{Bases Reducidas hp Greedy}

El nombre del método hp greedy viene de la combinación del \textit{``refinamiento p''} y del \textit{``refinamiento h"}. El \textit{refinamiento p} proviene de los métodos espectrales con bases polinomiales \cite{hesthaven_gottlieb_gottlieb_2007} y se refiere a la propiedad de que el error de representación disminuye al aumentar el grado del polinomio (en el caso de las bases reducidas aumenta el número de elementos en la base). Por otro lado el término de \textit{refinamiento h} se toma prestado de los métodos de diferencias finitas, donde el tamaño de cada celda de la grilla es representado por \textit{h}. En este caso el refinamiento ocurre en el espacio de los parámetros (y no en el dominio físico).

\subsection{Refinamiento h}

Partiendo de la siguiente notación:
\begin{itemize}
\item $V$: espacio de parámetros para un dado subdominio $D$.
\item $V_1, V_2$ : particiones de $V$.
\item $\Lambda_V$ : parámetros \textit{greedy} para $V$.
\item  $\hat{\Lambda}_{V}$: punto de anclaje para $V$.
%\item  $\hat{\Lambda}_{V_1}, \hat{\Lambda}_{V_2}$
\end{itemize}

El refinamiento en el dominio de los parámetros ocurre a partir de la división recursiva de cada subdominio $V$ del dominio total $D$ en dos subdominios $V_1$ y $V_2$. De forma que se obtiene una estructura de árbol binario.


Esta descomposición binaria del dominio está descrita en forma de pseudocódigo en el algoritmo \ref{alg:part}.

Al algoritmo ingresan tres objetos:

\begin{itemize}
\item $\lambda_V$: conjunto de parámetros resultado de un muestreo de $V$.
\item $\hat{\Lambda}_{V_1}, \hat{\Lambda}_{V_2}$: puntos de anclaje (son los primeros dos elementos de $\Lambda_V$).
\end{itemize}

Luego, para cada parámetro del conjunto $\lambda_V$ se evalúa su distancia a los puntos de anclaje a partir de la \textit{función de proximidad} $d: d(\lambda_1, \lambda_2)$:

\[
d(\lambda_1, \lambda_2) = ||\lambda_1 - \lambda_2||_2,
\]

de forma que se obtengan dos conjuntos; $\lambda_{V_1}$ con los $\lambda_{i}$ más proximos a $\hat{\Lambda}_{V_1}$, y $\lambda_{V_2}$ con los $\lambda_{i}$ más proximos a $\hat{\Lambda}_{V_2}$, tal que  $\lambda_{V} =\lambda_{V_1} \cup \lambda_{V_2}$. Este resultado es la división del espacio de parámetros a partir de los puntos de anclaje.

\begin{algorithm}
\caption{\texttt{Partition}\((\lambda_V, \hat{\Lambda}_{V_1}, \hat{\Lambda}_{V_2})\)}\label{alg:part}
\begin{algorithmic}[1]
\Require $\lambda_V, \hat{\Lambda}_{V_1}, \hat{\Lambda}_{V_2}$ 
\vspace{3mm}
\State $\lambda_{V_1} = \lambda_{V_2} = \emptyset$
\For{\textbf{each} $\lambda_i \in \lambda_V$}
	\If{$d(\lambda_i, \hat{\Lambda}_{V_1}) <d(\lambda_i, \hat{\Lambda}_{V_2})$}
		\State $\lambda_{V_1} = \lambda_{V_1} \cup \lambda_i$
	\ElsIf{$d(\lambda_i, \hat{\Lambda}_{V_1}) >d(\lambda_i, \hat{\Lambda}_{V_2})$}
		\State $\lambda_{V_2} = \lambda_{V_2} \cup \lambda_i$
	\Else
		\State $\lambda_{V'} = $ random choice$([\lambda_{V_1}, \lambda_{V_2}])$
		\State $\lambda_{V'} = \lambda_{V'} \cup \lambda_i$
	\EndIf
\EndFor
\vspace{3mm}
\Ensure $\lambda_{V_1}, \lambda_{V_2}$
\end{algorithmic}
\end{algorithm}

\subsection{Refinamiento hp-greedy}


El refinamiento hp-greedy es un método que combina el algoritmo greedy para la construcción de bases reducidas con la partición del dominio de parámetros. 

Esta partición recursiva del dominio de parámetros da lugar a una estructura de árbol binario, la cual tendrá diferentes niveles $l$ de profundidad, con un $l_{max}$ establecido por el usuario, de forma que $l : 0 \le l \le l_{max}$, donde $l =0$ es el nodo raíz. Cada nodo del árbol estará etiquetado por un conjunto de índices $B_l$, que parte de:

\[
B_0 = (0,),
\]
luego sus dos hijos ($l=1$) tendrán las etiquetas:
\[
B_1 = (0,0,) \ \ o \ \  (0, 1, ),
\]
y en general:
\[
B_l = (0, i_1, \cdots, i_l), \ \  con \ \ i_j = \{0, 1\},
\]

\noindent donde cada nivel $l$ tendrá un máximo de $2^l$ nodos.
Los nodos que no tengan hijos se llamarán nodos \textit{hojas}.
\begin{figure}[h!]
\centering
\includegraphics[width=.5\columnwidth]{Bl_lmax2.png}
\caption{Representación de los nodos de un árbol con $l_{max}=2$.}
\end{figure}

El método está explicado en el algoritmo \ref{alg:hp}; partiendo de un dado dominio de parámetros $V$ se construye una base reducida a partir de un conjunto de entrenamiento $\mathcal{T}_V = \{h_{\lambda_{V_i}}\}_{i=1}^{N}$, una \textit{tolerancia greedy} $\varepsilon$ y un $n_{max}$ (para esto se utiliza el algoritmo \textbf{[VERIFICAR REFERENCIA]}). Si el error de representación $\sigma$ es mayor que la tolerancia $\varepsilon$, y si la profundidad del nivel $l$ es menor a $l_{max}$, entonces se realizará una partición del dominio $V$ utilizando como puntos de anclaje a los dos primeros parámetros \textit{greedy}. En cada dominio se realizará el mismo procedimiento hasta que se cumpla que $l = l_{max}$ o hasta que $\sigma \le \varepsilon$.

 % El conjunto $\lambda_V$ representa el conjunto de parámetros de $\mathcal{T}_V$

\begin{algorithm}
\caption{\texttt{hpGreedy}\((\mathcal{T}_V, \lambda_V, \varepsilon, n_{max}, l, l_{max}, B_{l})\)}\label{alg:hp}
\begin{algorithmic}[1]
\Require $\mathcal{T}_V, \lambda_V, \varepsilon, n_{max}, l, l_{max}, B_{l}$ 
\vspace{3mm}
\State rb, $\Lambda_V$, $\sigma$ = \texttt{GreedyRB}($\mathcal{T}_V,\lambda_V, \varepsilon, n_{max}$) 
\vspace{3mm}
\If{$\sigma > \varepsilon$ \textbf{and} $l<l_{max}$}
	\State $\hat{\Lambda}_{V_1} = \Lambda_V[1]$
	\State $\hat{\Lambda}_{V_2} = \Lambda_V[2]$
	\State $\lambda_{V_1}, \lambda_{V_2} =$ \texttt{Partition}$(\lambda_V,\hat{\Lambda}_{V_1}, \hat{\Lambda}_{V_2})$
	\State $out_1 = $ \texttt{hpGreedy}\((\mathcal{T}_{V_1}, \lambda_{V_1}, \varepsilon, n_{max}, l+1, l_{max}, (B_{l}, 0))\)
	\State $out_2 = $ \texttt{hpGreedy}\((\mathcal{T}_{V_2} ,\lambda_{V_2}, \varepsilon, n_{max}, l+1, l_{max}, (B_{l}, 1))\)
	\State $out = out_1 \cup out_2$
\Else
	\State $out = \{( rb, \Lambda_V, B_l)\}$
\EndIf
\vspace{3mm}
\Ensure out
\end{algorithmic}
\end{algorithm}

El resultado del algoritmo \ref{alg:hp} es una estructura arbórea, donde cada nodo contiene la información de sus puntos de anclaje, por lo que en el caso de querer proyectar un conjunto de validación, cada onda gravitacional se proyectará a la base reducida del nodo hoja con el punto de anclaje más cercano al parámetro de la onda.


\subsection{Aplicación a Ondas Gravitacionales}

Se trabaja a partir de un conjunto de ondas gravitacionales con parámetro bidimensional, donde $\chi_{1_z} = \chi_{2_z} = \chi_z$, es decir que $\lambda = (q, \chi_z)$. De esta forma se puede graficar fácilmente el dominio de parámetros.



En la figura \ref{fig:part1} se puede observar una representación de la partición del dominio de parámetros. En la primera imagen se pueden ver los dos puntos de anclaje, que son los primeros dos elementos de la base global construida inicialmente. En cada nueva división se construye una nueva base global con la cual se realiza la siguiente partición de cada subdominio.


\begin{figure}[h!]
\centering
\includegraphics[width=1\columnwidth]{figs/particion2d.png}
\caption{Ejemplo de partición del espacio de parámetros bidimensional para $l_{max}= 1, 2, 3$ y $5$. En los primeros dos casos se muestran los puntos de anclaje.}
\label{fig:part1}
\end{figure}


En la figure \ref{fig:l0vl4} se compara el máximo error de representación obtenido para un conjunto de validación con una base global, es decir, con $l_{max} = 0$, y con $l_{max}=4$. La velocidad de convergencia es claramente mayor en el segundo caso.

\begin{figure}[h!]
\centering
\includegraphics[width=.8\columnwidth, trim={0, 1.3cm, 0, 1.4cm}]{figs/l0vsl4.pdf}
\caption{Base global ($l_{max} = 0$) versus base con $l_{max} = 4$} para distintos valores de $n$.
\label{fig:l0vl4}
\end{figure}


El aspecto más importante de este método es que permite disminuir la complejidad temporal del algoritmo a la hora de proyectar la base, a cambio de aumentar la complejidad espacial, pues si bien cada subdominio tendrá un máximo de $n_{max}$ elementos en su base, habrá un máximo de $2^{l_{max}}$ subdominios.

\begin{figure}[h!]
\centering
\includegraphics[width=.8\columnwidth ,trim={0, 1cm, 0, 1.2cm}]{figs/t_vs_nmax.pdf}
\caption{tiempos de proyección de un conjunto de validación a dos bases con distinto $l_{max}$ en función del $n_{max}$. En cada caso la linea de trazo representa el valor medio, y el área de color indica una desviación estandar desde el valor medio, para cada medición.}
\label{fig:t_vs_nmax}
\end{figure}


En la figura \ref{fig:t_vs_nmax} se graficó el tiempo de proyección de un conjunto de validación a dos bases \textit{hp-greedy} con distinto valor de $l_{max}$. Se observa que el tiempo es bastante lineal en relación al $n$ (elementos de las bases locales), y no parece ser afectado por $l_{max}$.



\begin{figure}[h!]
\centering
\includegraphics[width=.9\columnwidth, trim={0, 1cm, 0, 1.4cm}]{figs/nmax_lmax_t_grid.pdf}
\caption{Tiempo de proyección de un conjunto de validación para diferentes valores de $n_{max}$ y $l_{max}$}
\label{fig:t_grilla_nl}
\end{figure}


En la figura \ref{fig:t_grilla_nl} se puede ver el tiempo de proyección para más valores de $n_{max}$ y ${l_max}$. En los primeros valores de $l_{max}$ se observa un comportamiento similar al descrito anteriormente, donde el tiempo depende casi únicamente de $n_{max}$. Sin embargo al aumentar el $l_{max}$ se observa que el tiempo disminuye. Esto se puede entender en dos partes:

\begin{itemize}
\item \textbf{Independencia aparente entre el tiempo de proyección y $l_{max}$}: para realizar la proyección de cada onda del conjunto de validación en la base \textit{hp-greedy} primero se debe buscar el subdominio (la hoja) correspondiente utilizando los puntos de anclaje de la estructura arbórea de la base. Luego se proyectará la onda en la base local del subdominio en cuestión. Si bien la búsqueda en el árbol tiene una complejidad temporal $O(l_{max})$, el trabajo de cómputo más importante es el que se realizará al momento de proyectar la base, que es independiente de $l_{max}$, con una complejidad temporal $O(n_{max})$. Pero esto solo se cumple hasta ciertos valores de $l_{max}$.
\item \textbf{Disminución del tiempo de proyección al aumentar $l_{max}$}: ya se mencionó en más de una ocasión que por cada nivel $l$ hay un máximo de $2^l$ subdomínios. Es decir que si se quiere obtener el número de elementos de todas las bases en las hojas del árbol, suponiendo un árbol denso, este número será $n_{max} \times 2^{l_{max}}$. En la figura \ref{fig:t_grilla_nl} se utilizó un conjunto de entrenamiento con 1400 ondas, por lo que al llegar a unos valores de $l_{max} = 6$ y $n_{max} = 24$ en total debería haber 1536 elementos de base en total. Es decir, más elementos de base que ondas en el conjunto de entrenamiento. Por lo tanto al aumentar el $l_{max}$ rápidamente se aumenta el número de subdominios, reduciendo su tamaño como resultado y reduciendo el número de elementos de las bases locales (cada subdominio tendrá una cantidad de elementos menor a $n_{max}$). De esta forma se explica la disminución del tiempo de proyección para valores grandes de $l_{max}$, consecuencia del tamaño limitado del conjunto de entrenamiento.
\end{itemize}


\subsection{Hiperparámetros}

\begin{figure}[h!]
\centering
\includegraphics[width=1\columnwidth ,trim={2cm, 1cm, 1cm, 1.2cm}]{figs/overfit.pdf}
\caption{Ejemplos de \textit{sobreajuste}. A la izquierda variando $\varepsilon$ con $(n_{max}, l_{max}) = (25, 19)$ en un conjunto de parámetro unidimensional $(\lambda_i = q_i)$. A la derecha variando $l_{max}$ con $(n_{max}, \varepsilon) = (20, 1\times10^{-6})$ en un conjunto de parámetro bidimensional$(\lambda_{ij} = (q_i, \chi_{z_j})) $.}
\label{fig:overfit}
\end{figure}

Al momento de construir una base \textit{hp-greedy} entran en juego cuatro hiperparámetros. Los primeros tres son los parámetros de parada;
\begin{itemize}
\item $\bm{n_{max}}:$ determina la cantidad máxima de elementos para cada base local. A mayor cantidad de elementos el error de representación será menor, pero el tiempo requerido para proyectar un conjunto de validación a la base depende casi exclusivamente de este hiperparámetro.
\item $\bm{l_{max}}:$ determina la máxima profundidad de las hojas del árbol. En general al aumentar $l_{max}$ disminuye el error de representación, pero valores muy elevados junto a cierta combinación de hiperparámetros pueden dar lugar a sobreajustes en el modelo, un ejemplo de esto se puede ver en la figura \ref{fig:overfit}. Este es un comportamiento típico de las estructuras arbóreas.
\item $\bm{\varepsilon}$: la tolerancia \textit{greedy} interviene tanto en el tamaño de las bases locales como en la profundidad de las hojas del árbol. Un valor de $\varepsilon$ demasiado bajo también puede dar lugar a sobreajuste, sobre todo con valores muy altos de $l_{max}$. Un valor de $\varepsilon = 0$ implica que se obtiene un árbol totalmente denso, determinado únicamente por $n_{max}$ y $l_{max}$, y al aumentar el valor de $\varepsilon$ se puede pensar en la analogía de podar un árbol, de forma que se previene el sobreajuste.
\end{itemize}

Al cuarto hiperparámetro se le da el nombre de \textit{\textbf{semilla}} y se la denota con $\bm{\hat{\Lambda}_0}$;

\begin{figure}[h!]
\centering
\includegraphics[width=.8\columnwidth ,trim={1.1cm, 1.1cm, 1.2cm, 1.2cm}]{figs/Semillas_v_nmax_2D.pdf}
\caption{Error de validación para diferentes semillas.}
\label{fig:seeds0}
\end{figure}

\begin{itemize}
\item $\bm{\hat{\Lambda}_0}$: la semnilla no es más que el primer parámetro \textit{greedy} de la base global. En cada base local, el primer parámetro \textit{greedy} no es relevante, pero en el caso de las bases \textit{hp-greedy} cada semilla dará lugar a una división diferente del dominio de parámetros. En la figura \ref{fig:seeds0} se puede ver como cuatro semillas diferentes dan lugar a cuatro curvas de error con distinta convergencia. En la figura \ref{fig:seeds_part}, por otro lado, se observa el resultado de la partición del dominio para tres semillas diferentes. En general las semillas que mejor funcionan (con el conjunto de datos utilizado) son las que logran una partición regular del dominio de parámetros.
\end{itemize}



\begin{figure}[h!]
\centering
\includegraphics[width=1.05\columnwidth ,trim={1.1cm, 1cm, 1cm, 1.2cm}]{figs/3_semillas_particion.png}
\caption{Partición del espacio de parámetros para tres semillas diferentes; a la izquierda $\hat{\Lambda}_0 = (1.2 \ \ 0.38)$, a la derecha $\hat{\Lambda}_0 = (1.71 \ \ 0.04)$ y al centro $\hat{\Lambda}_0 =(4.55 \ \ -0.8)$.}
\label{fig:seeds_part}
\end{figure}




% Capitulo 3
\chapter{Optimización de Hiperparámetros}

\section{Planteo del Problema}

Sea $f: X \rightarrow \mathbb{R}$ una función que devuelve el máximo error de validación de un modelo entrenado a partir de una combinación de hiperparámetros $\textbf{x} \in X$, se desea encontrar $\hat{\textbf{x}}$:

\[
\hat{\textbf{x}} = arg \min_{\textbf{x} \in X} f(\textbf{x})
\]

Es decir, se busca encontrar la combinación óptima de hiperparámetros dentro de un dominio $X$ para obtener el mínimo error de representación en un dado conjunto de validación. En el caso de la construcción de una base \textit{hp-greedy} óptima: 

\[
\textbf{x} = (n_{max}, l_{max}, \varepsilon, \hat{\Lambda}_0).
\]


%El hiperparámetro de mayor interés es sin duda la $semilla$, pues a primera vista no hay una forma obvia de elegir su valor óptimo. Además una semilla óptima para una combinación de $(n_{max_1}, l_{max_1})$ no lo será necesariamente para otra combinación $(n_{max_2}, l_{max_2})$.

%También se observó que en ciertas situaciones un valor muy elevado de $l_{max}$ o muy pequeño de la $tolerancia$ $greedy$ pueden conducir a un modelo \textit{sobreajustado}. Por esto es de interés realizar una optimización que involucre a la totalidad de los hiperparámetros dentro del dominio $X$. 


El problema al momento de realizar esta optimización es que la función $f$ no tiene una expresión analítica, sino es que es el resultado de entrenar el modelo y evaluar el error de representación con un conjunto de validación, lo que la hace costosa de evaluar (computacionalmente hablando). Este capítulo se centrará principalmente en la \textbf{optimización Bayesiana} \cite{7352306, https://doi.org/10.48550/arxiv.1012.2599}, un método que intenta reducir al mínimo el número de evaluaciones de $f$ para encontrar $\hat{\textbf{x}}$ y se puede colocar dentro de una categoría llamada optimización secuencial basada en modelos, o \textbf{SMBO}\cite{dewancker2015bayesian,NIPS2011_86e8f7ab} (\textit{Secuential Model-Based Optimization}).


Además existen dos métodos muy utilizados que no utilizan modelos, los cuales son la \textbf{busqueda exaustiva} (o \textit{grid search}) y la \textbf{búsqueda aleatoria}. Estos métodos se utilizaron en casos sencillos de optimización para realizar una comparación con la optimización bayesiana.


\subsubsection*{Comentario sobre el dominio $X$}


Si bien la tolerancia \textit{greedy} $\varepsilon$ puede tomar cualquier valor real no nulo (a diferencia de $n_{max}, l_{max}$ y $\hat{\Lambda}_0$ que toman valores discretos), para simplificar la búsqueda de $\hat{\textbf{x}}$ se utilizaron siempre distribuciones discretas en el espacio logarítmico. Más específicamente se utilizaron conjuntos de la forma $C = \{1\times 10^{t} \  | \ a \leq t \leq b,  t \in {\mathbb{Z}} \}$. De esta forma $X$ será un conjunto finito y estará definido por los valores extremos de cada hiperparámetro. 


\section{Optimización Bayesiana}

\begin{figure}[h!]
\centering
\includegraphics[width=.7\columnwidth]{bayesian.png}
\caption{En la figura se observan tres iteraciones de una optimización bayesiana para una función sencilla con parámetro unidimensional. En linea punteada está representada la función real, mientras que con linea gruesa se representa el valor medio del modelo estadístico (en este caso construido utilizando procesos gaussianos). El área pintada en azul representa la incertidumbre del modelo, que tiende a cero en los puntos que representan las observaciones realizadas. Debajo se puede ver una función de adquisición en color naranja, que indica el siguiente punto a evaluar \cite{Feurer2019}.}
\label{fig:bayesian}
\end{figure}

La optimización bayesiana es un método que utiliza la información de todas las evaluaciones realizadas de la función $f$ para decidir que valor de $\textbf{x}$ evaluar a continuación, reduciendo así el número necesario de evaluaciones de $f$ para encontrar el mínimo.

Para explicar como funciona este método se parte de un formalismo llamado optimización secuencial basada en modelos, que no es más que una generalización de la optimización bayesiana.

\subsection{Optimización Secuencial Basada en Modelos}

La idea es aproximar la función $f$ a partir de un modelo sustituto $\mathcal{M}$.

Se parte de un conjunto de observaciones $D = \{(\textbf{x}^{(1)},y^{(1)}), \cdots, (\textbf{x}^{(k)},y^{(k)}) \}$, donde $y^{(j)} = f(\textbf{x}^{(j)})$, a partir del cual se ajusta el modelo sustituto $\mathcal{M}$. Luego utilizando las predicciones del modelo se maximiza una función $S$ llamada función de adquisición que elije el siguiente conjunto de hiperparámetros $\textbf{x}_i \in X$ para evaluar la función $f$ y se agrega el par $(\textbf{x}_i, f(\textbf{x}_i))$ al conjunto de observaciones $D$. Una vez hecho esto se vuelve a ajustar el modelo $\mathcal{M}$ y se repite el proceso, que está explicado en forma de pseudocódigo en el algoritmo \ref{alg:SMBO}.

%En la optimización secuencial basada en modelos, cada evaluación de la función $f$ se decide en base a un conjunto $D$ de observaciones realizadas previamente. Esto se logra entrenando un modelo sustituto $\mathcal{M}$ en base al conjunto $D$, y luego utilizando una función $S$ llamada función de \textit{adquisición}, que decidirá el mejor punto a evaluar en la función real.

%Se parte de un conjunto $D = {(\textbf{x}}$ generado a partir de un muestreo de observaciones de la función $f$ de la forma $(\textbf{x}_j, f(\textbf{x}_j))$, a partir del cual se ajusta el modelo sustituto $\mathcal{M}$. Luego, maximizando la función de adquisición $S$ se elije el siguiente conjunto de hiperparámetros $\textbf{x}_i$ para evaluar la función $f$ y se agrega el par $(\textbf{x}_i, f(\textbf{x}_i))$ al conjunto de observaciones $D$. Una vez hecho esto se vuelve a ajustar el modelo $\mathcal{M}$ y se repite el proceso, que está explicado en forma de pseudocódigo en el algoritmo \ref{alg:SMBO}.


\begin{algorithm}
\caption{\texttt{SMBO}}
\label{alg:SMBO}
\begin{algorithmic}[1]
\Require $f, X, S,\mathcal{M}$
\State $D =$ InicializarMuestras$(f, X)$
\vspace{1mm}
\For{$i = 1, 2, ...$}
	\State $\mathcal{M} =$ AjustarModelo$(D)$
	\State $\textbf{x}_{i} = arg \max_{\textbf{x}\in X} \mathcal{S}(\textbf{x}, \mathcal{M})$ .
	\State $y_i = f(\textbf{x}_i)$	\Comment{Paso costoso}
	\State $D = D \cup \{(\textbf{x}_i, y_i)\}$
\EndFor
\vspace{3mm}

\end{algorithmic}
\end{algorithm}

\subsection*{Optimización Bayesiana}

Lo que caracteriza a la optimización bayesiana dentro del formalismo de la optimización secuencial basada en modelos, es justamente la creación del modelo.
En la optimización bayesiana se construye un modelo estadístico, donde se representa con  $P(y|\textbf{x})$ la predicción del modelo, siendo $y$ el resultado de una evaluación $f(\textbf{x})$. El nombre del método se debe a que para la construcción del modelo se utiliza el teorema de Bayes:
  
 \[
 P(y|\textbf{x}) = \frac{P(\textbf{x}|y) \ P(y)}{P(\textbf{x})}
 \]
 
 En la terminología bayesiana, se conoce a $P(y|\textbf{x})$ como probabilidad a posteriorí o \textit{posterior}, que es proporcional a la probabilidad a priori o \textit{prior} $P(y)$ por la función de verosimilitud o \textit{likelihood} $P(\textbf{x}|y)$. La probabilidad $P(\textbf{x})$ es una probabilidad marginal que sirve como factor de normalización, por lo que no es de tanto interés.


\subsubsection*{Procesos Gaussianos}

\begin{figure}[h!]
\centering
\includegraphics[width=.8\columnwidth]{gaussian.png}
\caption{Proceso Gaussiano unidimensional con tres observaciones representadas por los puntos negros. La linea gruesa representa la media del modelo predictivo y la zona azul la varianza en cada caso. Se representa con linea de trazo las distribuciones normales para los valores $x_1, x_2,$ y $x_3$\cite{https://doi.org/10.48550/arxiv.1012.2599}.}
\label{fig:gaussian}
\end{figure}

Una opción muy utilizada para la construcción del \textit{prior} y actualización del \textit{posterior} son los procesos gaussianos. Una forma sencilla de entender un proceso gaussiano es pensarlo como una función que para cada valor de $x$ devuelve la media $\mu(x)$ y la varianza $\sigma(x)$ de una distribución normal, en el caso particular de que $x$ sea unidimensional (ver figura \ref{fig:gaussian}). Con $\textbf{x}$ multidimensional, se obtiene una distribución normal multivariable, caracterizada por el vector $\bm{\mu}(\textbf{x})$ y la matriz de covarianza $\Sigma(\textbf{x}, \textbf{x}')$.

Sin embargo en este trabajo no se utilizan procesos gaussianos, principalmente porque parten del supuesto de que $f$ es continua. Para una introducción a la optimización bayesiana con procesos gaussianos ver \cite{https://doi.org/10.48550/arxiv.1012.2599}.

\subsection{Mejora Esperada: Función De Adquisición} 
 
%Debido a que la función $f$ no será necesariamente convexa, y seguramente tendrá varios mínimos locales, es necesario mantener un equilibrio entre la explotación y la exploración, es decir, explorar los mínimos tentativos sin dejar de lado las zonas inexploradas, en donde podría haber algún mínimo por descubrir. De esto se encarga la función de adquisición.
Para la elección de los puntos a evaluar en la función real se maximiza la función de adquisición $S$. Existen varias propuestas de funciones de adquisición, pero en este caso se utiliza la \textbf{mejora esperada} o EI(\textit{Expected Improvement}) \cite{EI1}. Sea $y^*$ un valor de referencia, se define a la mejora esperada  con respecto a $y^*$ como:


\begin{equation}
\label{eq:ei_def}
EI_{y^*}(\textbf{x}) := \int_{-\infty}^{\infty} \max(y^*-y,0) p(y|\textbf{x}) \ dy
\end{equation}
 

\subsection{Estimador de Parzen con Estructura Arbórea} 

 El estimador de Parzen con estructura arbórea o \textbf{TPE} (\textit{Tree-Structured Parzen Estimator}) \cite{NIPS2011_86e8f7ab} es una estrategia que modela $P(x_i|y)$ para cada $x_i \in X_i$ (es decir, que $x_i$ representa a cada hiperparámetro por separado) a partir de dos distribuciones creadas a utilizando las observaciones $D$:

\begin{equation}
\label{eq:tpe}
P(x_i|y) =
	\begin{cases}
		\ell (x_i) & \text{si } y <y^{*} \\
		g(x_i) & \text{si } y \geq y^{*},
	\end{cases}
\end{equation}


Donde las densidades $\ell(x_i)$ y $g(x_i)$ se construyen a partir de dos conjuntos $D_{\ell}$ y $D_g$, ambos subconjuntos de $D$, tal que $D_{\ell}$ contiene todas las observaciones con $y < y^*$, y $D_g$ contiene a todo el resto de forma que $D = D_{\ell} + D_g$. El valor de referencia $y^{*}$ será un valor por encima del mejor valor observado de $f(\textbf{x})$, que se selecciona para ser un cuantil $\gamma \in (0,1)$ de los valores observados $y$ tal que $P(y<y^{*}) = \gamma$.


\subsubsection*{Mejora Esperada con TPE}
Aplicando la ecuación \eqref{eq:tpe} a la definición de mejora esperada \eqref{eq:ei_def} se obtiene la siguiente relación \cite{NIPS2011_86e8f7ab}:

\begin{equation}
EI_{y^*}(x_i) \propto \left( \gamma + (1-\gamma) \frac{g(x_i)}{\ell (x_i)} \right)^{-1}
\end{equation}

Es decir que para maximizar la mejora esperada se debe escoger un valor $x_i$ que maximice el cociente $\ell(x_i)/g(x_i)$ (o minimice $g(x_i)/\ell(x_i)$).

\subsubsection*{Estimación de las Densidades}

Las densidades de probabilidad se estiman utilizando ventanas de Parzen. Sea $D_x = \{x_i \ | \ (\textbf{x}, y) \in D_{\ell} \ (o\ D_g) \}$:

\begin{equation}
P(x_i) = \frac{\sum_{x_i'\in D_x}w_{x_i'}k(x_i, x_i') + w_p k(x_i, x_p) }{\sum_{x_i'\in D_x}w_{x_i'}+w_p},
\end{equation}
 

donde $w_{x_i'}$ es el peso de la observación $x_i'$(ver \cite{https://doi.org/10.48550/arxiv.1209.5111}), $x_p$ es un valor fijo \textit{prior}, $w_p$ es un peso \textit{prior} (por defecto igual a $1$) y $k$ es la función \textit{kernel}, que en este caso son distribuciones gaussianas truncadas centradas en los puntos $x_i'$ (ver \cite{10.1145/3377930.3389817} para más detalle).

%LEER https://tech.preferred.jp/en/blog/multivariate-tpe-makes-optuna-even-more-powerful/ !!!

\subsection*{Algoritmo}

Finalmente se puede ver el procedimiento completo del estimador de Parzen con estructura arbórea en el algoritmo \ref{alg:TPE}. Un detalle importante es que el algoritmo requiere un valor $n_c$, que es el número de candidatos que se utilizarán para maximizar el cociente $\ell(x_i)/g(x_i)$ (es decir, para maximizar la función de adquisición). En la linea 6 del algoritmo se realiza el muestro de los $n_c$ candidatos, utilizando la distribución $\ell(x_i)$, para luego seleccionar $x_i^*$ a partir del conjunto $C_i$.
Para este trabajo se utilizó la implementación de este algoritmo realizada en el paquete \textbf{\textit{Optuna}} \cite{optuna_2019} escrito en el lenguaje de programación Python.

\begin{algorithm}
\caption{\texttt{TPE}}
\label{alg:TPE}
\begin{algorithmic}[1]
\Require
\begin{tabular}{c}
$D = \{(\textbf{x}^{(1)},y^{(1)}), \cdots, (\textbf{x}^{(k)},y^{(k)}) \}$ \Comment{Observaciones}  \\ 
$n_t \in \mathbb{N}$ \Comment{número de iteraciones} \\ 
$n_c \in \mathbb{N}$ \Comment{número de candidatos} \\
$\gamma \in (0,1) $ \Comment{cuantil para obtener $y^*$}
\end{tabular} 

\vspace{1mm}
\For{$t = 1, 2, ..., n_t$}
	\State $D_{\ell} = \{ (\textbf{x}, y) \in D | y < y^*, \ \text{con } P(y<y^*) = \gamma  \}$
	\State $D_g = D - D_{\ell}$
	\For{$x_i = n_{max}, l_{max}, \varepsilon ,...$} \Comment{Para cada hiperparámetro}
		\State Construir $\ell(x_i)$ con $\{x_i \ | \ (\textbf{x}, y) \in D_{\ell}\}$ y $g(x_i)$ con $\{x_i \ | \ (\textbf{x}, y) \in D_g \}$.
		\State $C_i = \{ x_i^{(j)} \sim \ell(x_i)| j=1, ..., n_c \}$ \Comment{muestreo de $n_c$ candidatos para $x_i^*$}
		\State $x_{i}^* = arg \max_{x_i\in C_i} \ell(x_i)/g(x_i)$
	\EndFor
	\State $D = D \cup \{(\textbf{x}^*, f(\textbf{x}^*)\}$ \Comment{$\textbf{x}^*$ es el vector construido a partir de cada $x_i$.}
\EndFor
\Ensure $\textbf{x}$ con el mínimo valor $y$ en $D$.
\vspace{3mm}

\end{algorithmic}
\end{algorithm}
 
\section{Resultados}


\subsection{Conjunto pequeño: Comparación de métodos}

 La búsqueda exhaustiva o \textit{grid search} consiste en probar todas las combinaciones posibles dentro de un espacio de hiperparámetros para seleccionar la solución óptima. Es decir que si se quiere buscar la combinación óptima de $(n_{max}, l_{max})$ para un rango de valores $n_{max} \in N,$ $l_{max} \in L$ se deberán probar todas las combinaciones posibles del producto cartesiano $N \times L = \{(n_{max}, l_{max}) | n_{max} \in N, l_{max} \in L\}$. 
 
La ventaja de este método está en que el resultado óptimo está garantizado, pues se pueden comparar todos los resultados entre sí y elegir el mejor. El problema es que, como se mencionó, la función $f$ es costosa de evaluar, y por otro lado el número de combinaciones posibles escala exponencialmente con cada hiperparámetro extra a optimizar (además se debe tener en cuenta que la semilla $\hat{\Lambda}_0$ tendrá generalmente más de una dimension).

Por estas razones la búsqueda exhaustiva no será un método viable en la mayoría de los casos que son de interés para este trabajo. Sin embargo se puede poner a prueba con casos simplificados para luego comparar los resultados con otros métodos más eficaces.

\begin{figure}[h!]
\centering
\includegraphics[width=.8\columnwidth, trim={1.1cm, 1cm, 1cm, 1.2cm}]{grid_seeds_0.pdf}
\caption{Variación de la semilla $q_0$ para distintas combinaciones de $(n_{max}, l_{max})$}
\label{fig:grid_seed_0}
\end{figure}

Por ejemplo, para un conjunto de entrenamiento con cincuenta ondas equidistantes en el espacio del parámetro unidimensional $q: 1 < q < 8$, se quiere optimizar el error de representación para un conjunto de validación con mil ondas. Los hiperparámetros a optimizar son $\textbf{x} = (n_{max}, l_{max}, \hat{\Lambda}_0)$, dejando $\varepsilon$ fijo en $1\times 10^{-12}$ para simplificar la búsqueda, que se realiza en los siguientes intervalos:

\begin{align*}
n_{max} &\in [5, 20],\\
l_{max} &\in [1, 10],\\
\hat{\Lambda}_0 &\in \{q_0 \ | \ q_0 = 1 + i \Delta q, \ i\in \mathbb{N} : 0 \le i \le 49, \ \Delta q = 7/49 \}.
\end{align*}

Son 16 valores de $n_{max}$, 10 valores de $l_{max}$ y 50 para $q_0$ ($\hat{\Lambda}_0 = q_0$). Lo que hace un total de 8000 combinaciones posibles.



\begin{figure}[h!]
\centering
\includegraphics[width=1.05\columnwidth, trim={8cm, 1cm, 5cm, 1.1cm}]{grid_3_seeds.pdf}
\caption{Máximo error de validación en función de $n_{max}$ y $l_{max}$ para tres diferentes semillas $q_0$. }
\label{fig:grid_3_seeds}
\end{figure}

Si bien no se puede graficar el error en función de los tres hiperparámetros a la vez, se puede obtener bastante información al dejar fijo uno o dos hiperparámetros. Por ejemplo en la figura \ref{fig:grid_seed_0} se ven los resultados de variar únicamente la semilla para diferentes combinaciones de $n_{max}$ y $l_{max}$. Se observa que en este conjunto de datos la semilla suele ser óptima a valores cercanos a $q_0 = 8$, pero a la vez al aumentar el $n_{max}$ la influencia de la semilla es menor; con $(n_{max}, l_{max})=(20, 4)$ hay una diferencia de un orden de magnitud entre la mejor y la peor semilla, mientras que con $(n_{max}, l_{max})=(10, 9)$, por ejemplo, hay una diferencia de diez ordenes de magnitud.

Luego, en la figura \ref{fig:grid_3_seeds} se observa el error de validación en función de las combinaciones posibles de  $n_{max}$ y $l_{max}$ para tres diferentes semillas. A la derecha, con $q_0=8$ se observa el mejor error de representación obtenido en la búsqueda exhaustiva, con un valor de $4\times10^{-30}$ (el cual se obtuvo para 16 combinaciones de hiperparámetros), casi 15 ordenes de magnitud menor que el siguiente mejor error.

\begin{figure}[h!]
\centering
\includegraphics[width=.9\columnwidth, trim={1cm, 1cm, 1cm, 1.1cm}]{benchmark_0.pdf}
\caption{Comparación de convergencia. Se muestran los cuartiles para 20 optimizaciones realizadas, en cada caso.}
\label{fig:bench0}
\end{figure}


\subsection{Semilla 2D: Optimización Completa}




\begin{figure}[h!]
\centering
\includegraphics[width=.9\columnwidth, trim={1cm, 1cm, 1cm, 1.1cm}]{opt_1d_value.pdf}
\caption{caption }
\label{fig:optuna_1_value}
\end{figure}

  

\begin{figure}[h!]
\centering
\includegraphics[width=1.05\columnwidth, trim={10cm, 1cm, 10cm, 1.1cm}]{opt_1d_params.pdf}
\caption{cption }
\label{fig:optuna_1_params}
\end{figure}



\begin{figure}[p!]
\centering
\includegraphics[width=1\columnwidth, trim={6cm, 5cm, 12cm, 5cm}]{Optuna_2D_noepsilon.pdf}
\caption{caption }
\label{fig:optuna_2d}
\end{figure}

\begin{figure}[p!]
\centering
\includegraphics[width=1\columnwidth, trim={6cm, 5cm, 12cm, 5cm}]{Optuna_2D.pdf}
\caption{caption }
\label{fig:optuna_2d}
\end{figure}



\subsection{Importancia de los Hiperparámetros}
\chapter{Resultados}

En este capítulo se recogen los resultados más relevantes de la optimización de Hiperparámetros
\section{Optimización del Máximo Error de Validación}
En esta sección se encuentran los resultados de las optimizaciones realizadas teniendo en cuenta únicamente el máximo error de validación
\subsection{Conjunto pequeño: Comparación de métodos}


\begin{figure}[h!]
\centering
\includegraphics[width=.9\columnwidth, trim={1cm, 1cm, 1cm, 1.1cm}]{benchmark_0.pdf}
\caption{Comparación de convergencia. Se muestran los cuartiles para 20 optimizaciones realizadas, en cada caso.}
\label{fig:bench0}
\end{figure}

Utilizando un conjunto de entrenamiento con cien ondas equidistantes en el espacio del parámetro unidimensional $q: 1 < q < 8$, se quiere optimizar el error de representación para un conjunto de validación con quinientas ondas (cinco veces más denso). Los hiperparámetros a optimizar son $\textbf{x} = (n_{max}, l_{max}, \hat{\Lambda}_0)$, dejando $\varepsilon$ fijo en $1\times 10^{-12}$ para simplificar la búsqueda, que se realiza en los siguientes intervalos:

\begin{align*}
n_{max} &\in \{5, 6, 7, ..., 20\},\\
l_{max} &\in \{1, 2, 3, ..., 10 \},\\
\hat{\Lambda}_0 &\in \{q_0 \ | \ q_0 = 1 + i \Delta q, \ i\in \mathbb{N} : 0 \le i \le 99, \ \Delta q = 7/99 \}.
\end{align*}

Son 16 valores de $n_{max}$, 10 valores de $l_{max}$ y 100 para $q_0$ ($\hat{\Lambda}_0 = q_0$). Lo que hace un total de 16000 combinaciones posibles.

En la figura \ref{fig:bench0} se puede ver el resultado de realizar 20 optimizaciones de 100 iteraciones con tres diferentes métodos; búsqueda aleatoria (\textit{random}), TPE y TPE multivariante (los tres métodos están implementados en Optuna \cite{optuna_2019}). En linea oscura se representa la media del mejor error a cada iteración, y la zona sombreada representa los cuartiles. Aparte, en linea de trazo se marca el mejor error obtenido realizando una búsqueda exhaustiva (\textit{grid search}).


En las primeras 10 repeticiones los tres métodos son equivalentes, pues para el algoritmo TPE (tanto el normal como el multivariante) se parte de un muestreo aleatorio de 10 observaciones. En la figura \ref{fig:bench0} se ve claramente que luego de las décima iteración se produce el cambio más notorio entre los tres métodos. Gráficamente se puede ver que ambas versiones del algoritmo TPE dan un mejor resultado que la búsqueda aleatoria, pero aparte de esto se pueden utilizar métricas como el \textbf{mejor valor encontrado} para la mediana de $y$ o el \textbf{área bajo la curva} o \textbf{AUC} (\textit{Area Under the Curve})\cite{Dewancker2016ASF}. En la tabla \ref{tab:comp1} están los resultados de estas métricas, considerando solo las últimas 90 iteraciones.


\begin{table}
\centering
\begin{tabular}{@{}lccc@{}}
\toprule
\textbf{Algoritmo} & AUC (Mediana) & Mejor $Mediana(y)$ encontrada \\ 
\midrule
Búsqueda Aleatoria & $2.59\times 10^{-10}$ & $2.26\times 10^{-13}$ \\ 
TPE & $1.20\times 10^{-11}$ & $1.04\times 10^{-13}$ \\ 
TPE Multivariante & $5.20\times 10^{-12}$ & $5.48\times 10^{-14}$ \\ 
\bottomrule
\end{tabular} 
\caption{Comparación entre algoritmos de optimización.}
\label{tab:comp1}
\end{table}

\subsubsection*{Búsqueda Exhaustiva}
La búsqueda exhaustiva o \textit{grid search} consiste en probar todas las combinaciones posibles dentro de un espacio de hiperparámetros para seleccionar la solución óptima. Es decir que si se quiere buscar la combinación óptima de $(n_{max}, l_{max})$ para un rango de valores $n_{max} \in N,$ $l_{max} \in L$ se deberán probar todas las combinaciones posibles del producto cartesiano $N \times L = \{(n_{max}, l_{max}) | n_{max} \in N, l_{max} \in L\}$. 

\begin{figure}[h!]
\centering
\includegraphics[width=1.05\columnwidth, trim={8cm, 1cm, 5cm, 1.1cm}]{grid_3_seeds.pdf}
\caption{Máximo error de validación en función de $n_{max}$ y $l_{max}$ para tres diferentes semillas $q_0$. }
\label{fig:grid_3_seeds}
\end{figure}

\begin{figure}[h!]
\centering
\includegraphics[width=.8\columnwidth, trim={1.1cm, 1cm, 1cm, 1.2cm}]{grid_seeds_0.pdf}
\caption{Máximo error de validación en función de la semilla $q_0$ para distintas combinaciones de $(n_{max}, l_{max})$}
\label{fig:grid_seed_0}
\end{figure}


 
%La ventaja de este método está en que el resultado óptimo está garantizado, pues se pueden comparar todos los resultados entre sí y elegir el mejor. El problema es que, como se mencionó, la función $f$ es costosa de evaluar, y por otro lado el número de combinaciones posibles escala exponencialmente con cada hiperparámetro extra a optimizar (además se debe tener en cuenta que la semilla $\hat{\Lambda}_0$ tendrá generalmente más de una dimension).

%Por estas razones la búsqueda exhaustiva no será un método viable en la mayoría de los casos que son de interés para este trabajo. Sin embargo se puede poner a prueba con casos simplificados para luego comparar los resultados con otros métodos más eficaces.

Si bien no se puede graficar el error en función de los tres hiperparámetros a la vez, se puede obtener bastante información al dejar fijo uno o dos hiperparámetros. Por ejemplo en la figura \ref{fig:grid_3_seeds} se observa el error de validación en función de las combinaciones posibles de  $n_{max}$ y $l_{max}$ para tres diferentes semillas. 

Luego en la figura \ref{fig:grid_seed_0} se ven los resultados de variar únicamente la semilla para diferentes combinaciones de $n_{max}$ y $l_{max}$. 
En este conjunto de datos se observa que para $(n_{max}, l_{max})=(20, 10)$ hay una diferencia de 9 ordenes de magnitud entre la peor y la mejor semilla (datos en color verde). Sin embargo para $(n_{max}, l_{max})=(15, 10)$ la diferencia es de 13 ordenes de magnitud (datos de color azul). Además el valor óptimo de la semilla no coincide exactamente en estos dos ejemplos, aunque tengan un comportamiento similar. Es decir que la influencia de la semilla depende del resto de hiperparámetros, sobre todo se tiene que tener en cuenta que en este caso la tolerancia \textit{greedy} tenía un valor $\varepsilon = 1\cdot 10^{-12}$, por lo que no se va a obtener un resultado mucho mejor que este.

\begin{figure}[h!]
\centering
\includegraphics[width=1\columnwidth, trim={5cm, 5cm, 11cm, 5cm}]{Optuna_1d_simple.pdf}
\caption{Optimización con 100 iteraciones utilizando el algoritmo TPE multivariante para el conjunto de entrenamiento con semilla unidimensional.}
\label{fig:optuna_1d}
\end{figure}

\subsubsection*{Tiempo de Optimización}

Si bien la búsqueda exhaustiva garantiza encontrar el mejor resultado posible dentro del espacio de búsqueda, el tiempo necesario para realizar la búsqueda hace que el método no sea aplicable a casos relativamente complejos. En este caso sencillo, con 16000 combinaciones, la búsqueda requirió \textbf{25 horas} para completarse. En cambio las optimizaciones realizadas utilizando los algoritmos TPE tardaron una media de \textbf{8 minutos}.

Por último en la figura \ref{fig:optuna_1d} se puede ver gráficamente el proceso de optimización utilizando el algoritmo TPE multivariable.

\subsection{Optimización Completa}

\begin{figure}[p!]
\centering
\includegraphics[width=1\columnwidth, trim={6cm, 5cm, 12cm, 5cm}]{Optuna_2D.pdf}
\caption{Optimización con 500 iteraciones para semilla de dos dimensiones utilizando el algoritmo TPE Multivariante con 4 trabajadores en paralelo.}
\label{fig:optuna_2d}
\end{figure}

Utilizando un conjunto de entrenamiento con 70 valores discretos de $q$ equidistantes en el rango [1, 8] y 20 de $\chi_z$ ($\chi_{z_1} = \chi_{z_2}$) en el rango [-0.8, 0.8], dando lugar a un total de 1400 funciones de onda, se muestran los resultados de optimizar el máximo error de validación utilizando un conjunto de validación con 100 valores para $q$ y 30 valores para $\chi_z$ con un total de 3000 funciones de onda.

La optimización se realizó en los siguientes espacios de búsqueda:
 
\begin{align*}
n_{max} &\in \{10, 11, 12, ..., 60\},\\
l_{max} &\in \{2, 3, 4, ..., 20 \},\\
\varepsilon &\in \{ 10^{-20}, 10^{-19}, 10^{-18}, ..., 10^{-4}\},\\
Q_0 &= \{ q_0 \ | \ q_0 = 1 + i \Delta q, \ i\in \mathbb{N} : 0 \le i \le 69, \ \Delta q = 7/69 \},\\
X_0 &= \{\chi_{z_0} \ | \ \chi_{z_0} = -0.8 + j \Delta \chi_z, \ j\in \mathbb{N} : 0 \le j \le 19, \ \Delta \chi_z = 1.6/19 \}, \\
\hat{\Lambda}_0 &\in \{ (q_0, \chi_{z_0}) \ | \ q_0 \in Q_0,\ \chi_{z_0} \in X_0 \}.
\end{align*}


En total se realizaron 500 iteraciones, y el mejor máximo error de validación obtenido en la iteración número 269 fue de $1{.}45\times 10^{-6}$ con los hiperparámetros:

\begin{align*}
&n_{max}^* = 59, \\
 &l_{max}^* = 4, \\
&\varepsilon^* = 10^{-17},\\
 &q_0^* = 7.899, \\
 &\chi_{z_0}^* = 0.716.
\end{align*}

En la figura \ref{fig:optuna_2d} se observa la evolución de la optimización realizada, que requirió alrededor de 8 horas para completarse. Se puede ver que para cada hiperparámetro se observa una convergencia a cierto valor, pero sin dejar de lado la exploración, es decir, que se siguen evaluando hiperparámetros fuera del rango que parece óptimo, de forma que se evita caer mucho tiempo en mínimos locales.


\subsubsection*{Error de Prueba}

Luego de realizar la optimización se puede utilizar un conjunto de prueba, es decir, un conjunto de elementos que no se usaron para el entrenamiento ni para la optimización, para poner a prueba el resultado obtenido.

En este caso, utilizando un conjunto de prueba con 6000 elementos como resultado de un muestreo de 150 valores para $q$ y 40 para $\chi_z$, se calculó el máximo error relativo, siendo:

\[
error \ relativo = \frac{\| h_{\lambda} - P_n h_{\lambda} \|^2}{\| h_{\lambda} \|^2}
\]

Como resultado, se obtuvo un máximo error relativo de $3.84 \cdot 10^{-7}$.



\subsubsection{Importancia de los Hiperparámetros}

Una vez realizada una optimización se puede estimar la importancia relativa de cada hiperparámetro con el algoritmo fANOVA \cite{pmlr-v32-hutter14}. Básicamente la idea es dividir la varianza total en distintos componentes que representen la varianza producida por cada hiperparámetro. En la figura \ref{fig:param_import} se observa a la izquierda en naranja los resultados para la optimización realizada, y a la derecha en azul se ven los resultados para otro espacio de búsqueda, esta vez con $n_{max} \in \{20, ..., 30\}$  y $l_{max} = \{2, ..., 8\}$ (es decir que se redujo el espacio de búsqueda para $n_{max}$ y $l_{max}$). Se puede ver que la importancia que tienen los hiperparámetros depende claramente del espacio de búsqueda, pero en general se observó que $n_{max}$ y $l_{max}$ suelen tener mayor importancia relativa al resto de hiperparámetros.

\begin{figure}[h!]
\centering
\includegraphics[width=1\columnwidth, trim={5cm, 2cm, 5cm, 2cm}]{params_importance_full.pdf}
\caption{Importancia relativa de los hiperparámetros para dos espacios de búsqueda diferentes. $n_{max}$ y $l_{max}$ suelen ser los hiperparámetros más importantes en la mayoría de los espacios de búsqueda.}
\label{fig:param_import}
\end{figure}




\section{Optimización Multiobjetivo}

En esta sección se encuentran los resultados más relevantes de la optimización multiobjetivo, que optimiza el máximo error de evaluación al mismo tiempo que el tiempo necesario para proyectar el conjunto de validación a la base creada.

\subsection{Frente de Pareto}

Utilizando un conjunto de entrenamiento con mil funciones de ondas equidistantes en el espacio del parámetro unidimensional $q: 1 < q < 8$, y un conjunto de validación diez veces más denso (diez mil funciones de onda) se optimizó $\textbf{x} = (n_{max}, l_{max},\varepsilon, \hat{\Lambda}_0)$, en los siguientes intervalos:

\begin{align*}
n_{max} &\in \{10, 11, 12, ..., 20\},\\
l_{max} &\in \{1, 2, 3, ..., 10 \},\\
\varepsilon &\in \{  10^{-20}, 10^{-19}, 10^{-18}, ..., 10^{-6}\}, \\
\hat{\Lambda}_0 &\in \{q_0 \ | \ q_0 = 1 + i \Delta q, \ i\in \mathbb{N} : 0 \le i \le 999, \ \Delta q = 7/999 \}.
\end{align*}

Una buena forma de visualizar los resultados es graficando ambos objetivos a la vez. En la figura \ref{fig:pareto} cada punto representa una observación, y lo interesante de este gráfico es que puede observar fácilmente el frente de Pareto (en color naranja). El frente de Pareto está conformado por aquellas observaciones que no sean dominadas por ninguna otra, pero son incomparables entre sí, por lo que este conjunto reemplaza a $\textbf{x}^*$. Una vez obtenido este conjunto se debe elegir que objetivo es más importante y seleccionar una configuración $\textbf{x}$ dentro del conjunto de Pareto.

\begin{figure}[h!]
\centering
\includegraphics[width=.8\columnwidth, trim={1cm, 1cm, 1cm, 1cm}]{motpe_pareto.pdf}
\caption{Máximo error de validación versus el tiempo de proyección. Cada punto representa una observación. En color naranja se marca el frente de Pareto.}
\label{fig:pareto}
\end{figure}

\subsection{Tiempo de Proyección versus Hiperparámetros}
\label{sec:corr_t}

\begin{figure}[p]
\centering
\includegraphics[width=1\columnwidth, trim={5cm, 2cm, 5cm, 2cm}]{motpe_correlation.pdf}
\caption{Relación entre el tiempo de ejecución y los diferentes hiperparámetros para una optimización de 160 iteraciones.}
\label{fig:motpe_param_rel}
\end{figure}


La optimización multiobjetivo resulta muy interesante, pero en el contexto de ondas gravitacionales, se observa que hay una gran dependencia entre el tiempo de proyección y $n_{max}$. Mas bien, como ya se observó en la figura \ref{fig:t_vs_nmax} de la sección \ref{sec:hp-gw} sobre bases reducidas \textit{hp-greedy}, el tiempo de proyección depende casi exclusivamente del valor de $n_{max}$, siempre que el conjunto de entrenamiento sea lo suficientemente denso ($n_{max} \cdot 2^{l_{max}} < N$).

En la figura \ref{fig:motpe_param_rel} se puede ver la dependencia entre los diferentes hiperparámetros y el tiempo de proyección en segundos, para las observaciones realizadas. Claramente hay una tendencia bastante lineal al considerar $n_{max}$.

Para entender la dependencia entre $n_{max}$ y el tiempo de proyección hay que recordar la ecuación \eqref{eq:coefs} de los coeficientes $c_{i, \lambda}$:

\[
c_{i, \lambda} = \langle e_i, h_{\lambda} \rangle .
\]

Para proyectar una función de onda $h_{\lambda}$ a una base reducida se deben calcular los coeficientes $c_{i, \lambda}$, y el número de coeficientes será igual al número de elementos de la base $\{  e_i\}$. Por lo tanto mientras más elementos tengan las bases locales, mayor será el tiempo de proyección.

\chapter{Conclusión}

Las bases reducidas con refinamiento \textit{hp-greedy} mostraron una clara ventaja con respecto a las bases reducidas globales. Para obtener un mismo error máximo de validación, una base local necesita muchos más elementos, lo que se traduce en un mayor tiempo requerido para proyectar un espacio.

El problema de las bases reducidas \textit{hp-greedy}, que no está presente en la construcción de una base reducida global, es la complejidad añadida debido al gran número de configuraciones de hiperparámetros que afectan al rendimiento final. 

%La optimización de hiperparámetros es un problema que ha tenido un gran desarrollo en los últimos años debdio a la necesidad de construir modelos cada vez más complejos dentro del área de aprendizaje supervisado y aprendizaje profundo.

En este trabajó se utilizaron métodos de optimización bayesiana para resolver el problema de la selección de hiperparámetros, y se lograron resultados que cumplieron con las espectativas iniciales. Se observó que para los hiperparámetros más relevantes se necesitaban relativamente pocas iteraciones hasta que alcanzar una convergencia.

También se puso a prueba la optimización por búsqueda aleatoria, con buenos pero inferiores resultados, y se mostró que la búsqueda exhaustiva, si bien garantiza un resultado óptimo, requeriría un tiempo prohibitivo para ser puesta en práctica en un escenario realista.


Por últimoi,una opción interesante fue la optimización multiobjetivo, que permitió optimizar el error de representación al mismo tiempo que el tiempo de proyección. Sin embargo la gran correlación que hay entre el segundo objetivo y $n_{max}$ hace que el aporte de este método no esté a la altura de la complejidad extra que surge del mismo. Pues para limitar el tiempo de proyección simplemente se limita el máximo valor de $n_{max}$ dentro de la optimización a realizar.




\appendix
%\include{apend1}

\begin{biblio}
\bibliography{mibib}
\end{biblio}


\begin{postliminary}

%\begin{seccion}{Publicaciones asociadas}
 % \begin{enumerate}
  %\item Mi primer aviso en la revista \textbf{ABC}, 1996
%  \item Mi segunda publicaci\'{o}n en la revista \textbf{ABC}, 1997
%  \end{enumerate}
%\end{seccion}

\begin{seccion}{Agradecimientos}
Agradezco a mi familia, por darme las condiciones y el apoyo para hacer una carrera universitaria. 
A mis compañeros, que fueron mi ejemplo a seguir y de quienes nunca dejé de aprender cosas nuevas. 
A mi director Manuel Tiglio, y mi codirector Carlos Figueroa, por haberme dado la oportunidad de trabajar en esta tesis. 
Y a mis amigos, por siempre estár ahí. 
\end{seccion}

\end{postliminary}

\end{document}

