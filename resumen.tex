\begin{resumen}%


La inferencia de parámetros para una colisión binaria de agujeros negros es un área de gran importancia dentro de la ciencia de ondas gravitacionales, sobre todo con el trabajo conjunto de los interferómetros LIGO, VIRGO y KAGRA generando una gran cantidad de datos para ser analizados. El estándar actual para realizar esta inferencia requiere la producción de funciones de onda en tiempo real, lo que no es posible utilizando relatividad numérica. Los modelos sustitutos de orden reducido son una gran alternativa que permite generar resultados precisos en el orden de los milisegundos.

En esta tesis se trató el problema de la optimización de hiperparámetros para un sistema de aprendizaje supervisado, el cual consiste en la primera etapa de la construcción de un modelo sustituto. Este sistema de aprendizaje es un refinamiento del método de las bases reducidas, ya utilizado en la construcción de modelos sustitutos. 

Para la optimización se utilizaron métodos bayesianos, los cuales se vienen utilizando bastante dentro de la ciencia de datos en los últimos años debido a la necesidad de crear modelos cada vez más complejos y precisos. Esta optimización se realizó utilizando la librería llamada Optuna escrita en Python. Se comparó el método con la búsqueda aleatoria y se obtuvieron buenos resultados, mostrando una clara superioridad de la optimización bayesiana, por lo menos en este contexto.


\end{resumen}

\begin{abstract}%
Parameter inference for binary black hole collisions is an area of great importance within gravitational wave science, especially with the joint work of the LIGO, VIRGO and KAGRA interferometers generating a large amount of data to be analyzed. The current standard for performing this inference requires the production of wave functions in real time, which is not possible using numerical relativity methods. Reduced order surrogate models are a great alternative that allows the generation of accurate results in the order of milliseconds.

In this thesis the problem of hyperparameter optimization for a supervised learning system is treated, which consists of the first stage of the construction of a surrogate model. This learning system is a refinement of the reduced basis method, already used in the construction of surrogate models. 

Bayesian methods were used for the optimization, which have been widely used in data science in recent years due to the need to create increasingly complex and accurate models. This optimization was performed using the Optuna library written in Python. The method was compared with random search and good results were obtained, showing a clear superiority of Bayesian optimization, at least in this context.

\end{abstract}


%%% Local Variables: 
%%% mode: latex
%%% TeX-master: "template"
%%% End: 
